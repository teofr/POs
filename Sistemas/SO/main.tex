\documentclass[presentation]{beamer}

\mode<handout>{}
\usepackage[utf8]{inputenc}
\usepackage{hyperref}
\usepackage{algorithm} 
\usepackage[noend]{algpseudocode}
\usepackage[spanish]{babel}
\usepackage{listings}
\usepackage{ulem}


\title{Prueba de Oposición}
\subtitle{AY2 - Sistemas}
\author{\large{Teodoro Freund}}
\date{20 de Septiembre del 2017 \\ \footnotesize{16:00}} %Chequear

\beamertemplatenavigationsymbolsempty

\begin{document}

\floatname{algorithm}{Algoritmo}
\floatname{Procedure}{pro}
\begin{frame}
\titlepage
\end{frame}

\begin{frame}{Contexto}
\begin{itemize}
\item Materia: Sistemas Operativos
\pause
\item Práctica 3: Sincronización entre procesos
\pause
\item Primera Parte:
\begin{itemize}
\item Procesos y API del SO
\item Scheduling
\pause
\item \textbf{Sincronización entre procesos}
\pause
\end{itemize}
\item Segunda Parte:
\begin{itemize}
\item Administración de Memoria

\item Administración de Entrada/Salida
\item Protección y Seguridad
\item Sistemas de Archivos
\item Sistemas Distribuidos
\item Programación Concurrente
\end{itemize}
\end{itemize}

\end{frame}

\begin{frame}[fragile]{El Ejercicio 21: La cena de los antropófagos}

Una tribu de antropófagos cena de una gran cacerola que puede contener M porciones de misionero asado. Cuando un antropófago quiere comer se sirve de la cacerola, excepto que esté vacía. Si la cacerola está vacía, el antropófago despierta al cocinero y espera hasta que éste rellene la cacerola.\\


Pensar que, sin sincronización el antropófago hace:



\begin{lstlisting}[language=C++]
while(true){
  obtener_porcion();
  comer();
}
\end{lstlisting}

La idea es que el antropófago no pueda comer si la cacerola está vacía y que el cocinero sólo trabaje si está vacía la cacerola.

\end{frame}





\begin{frame}{¿Por qué?}

\begin{itemize}
\item Es un buen caso de ejemplo de la primitiva menos usada: condición de variable.

\item Si bien entra dentro del marco Productor-Consumidor, tiene un giro sobre el mismo.

\item Da para discutir distintas implementaciones y sus ventajas y desventajas.

\item Por todo esto, es un buen ejercicio para dar en clase.


\end{itemize}

\end{frame}

\begin{frame}{Pensemos partes de la solución}

\begin{itemize}
\item El cocinero espera a que la cacerola esté vacía
\pause

\item El antropófago, a que esté llena
\pause
\item Esas condiciones podrían ser señalizadas de alguna manera en vez de preguntadas por dichos hilos
\pause
\item Necesitamos exclusión mútua sobre la 'cacerola'
\end{itemize}

\end{frame}


\begin{frame}[fragile]{Primera Versión \footnote{Sacada de una clase de IIT}}

\lstset{language=C++, basicstyle=\scriptsize}

\begin{columns}
 
\column{0.5\textwidth}
\begin{lstlisting}
//Variables globales
int porciones = 0;
mutex = semaforo(1);
vacia = semaforo(0);
lleno = semaforo(0);


void* cocinero(void* s){
  while(true){
    vacia.wait();
    porciones += llenarCacerola();
    llena.signal();
  }
}
\end{lstlisting}
 
\column{0.5\textwidth}
\begin{lstlisting}
void* antropofago(void* s){
  while(true){
    mutex.wait();
    if (porciones == 0){
      vacia.signal();
      lleno.wait();
    }
    porciones--;
    obtener_porcion();
    mutex.signal();
    comer();
  }
}
\end{lstlisting}
\end{columns}





\end{frame}

\begin{frame}{Pros y contras de esta solución}
\underline{Pros:}
\begin{itemize}
\item<1-> Resuelve el problema
\item<2-> No hay busy-wait
\item<3-> \only<3>{Mantiene exclusión mútua sobre \texttt{obtener\_porcion()}} \only<4->{\sout{Mantiene exclusión mútua sobre \texttt{obtener\_porcion()}}}
\end{itemize}

\only<4->{\underline{Contras:}}
\begin{itemize}
\item <4-> Mantiene exclusión mútua sobre \texttt{obtener\_porcion()}
\item<5-> Innumerables llamadas al sistema operativo
\item<6-> La abstracción está lejos de ser ideal
\end{itemize}

\end{frame}


\begin{frame}[fragile]{Segunda versión}

\lstset{language=C++, basicstyle=\scriptsize}

\begin{columns}

\column{0.5\textwidth}
\begin{lstlisting}
//Variables globales
int porciones = 0;
vacia = condicion();
llena = condicion();
condMutex = semaforo(1);

void* cocinero(void* s){
  while(true){
    condMutex.lock();
    while(porciones > 0){
      vacia.wait(condMutex);
    }
    porciones += llenarCacerola();
    llena.broadcast();
    condMutex.unlock();
  }
}
\end{lstlisting}

\column{0.5\textwidth}

\begin{lstlisting}
void* antropofago(void* s){
  while(true){
    condMutex.lock();
    while(porciones <= 0){
      vacia.broadcast();
      llena.wait(condMutex);
    }
    int porcion = porciones--;
    condMutex.unlock();  
    obtener_porcion();
    comer();    
  }
}
\end{lstlisting}


\end{columns}

\end{frame}

\begin{frame}{Pros y contras de esta solución}
\underline{Pros:}
\begin{itemize}
\item<1-> Resuelve el problema
\item<2-> No hay busy-wait
\item<3-> No mantiene exclusión mútua sobre \texttt{obtener\_porcion()}
\item<4-> Las primitivas parecen abstraer mejor la situación
\end{itemize}

\only<5->{\underline{Contras:}}
\begin{itemize}
\item<5-> Innumerables llamadas al sistema operativo \only<6-> {\emph{¿Esto es algo malo?}}
\end{itemize}

\end{frame}





\begin{frame}[fragile]{Tercera versión, spinning}

\lstset{language=C++, basicstyle=\scriptsize}

\begin{columns}

\column{0.5\textwidth}
\begin{lstlisting}
//Variables globales
int porciones = 0;
sLock = spinLock();

void* cocinero(void* s){
  while(true){
    while(porciones > 0){}
    sLock.lock();
    porciones += llenarCacerola();
    sLock.unlock();
  }
}
\end{lstlisting}

\column{0.5\textwidth}

\begin{lstlisting}
void* antropofago(void* s){
  while(true){
    while (porciones == 0){}
    sLock.lock();
    if (porciones >= 0){
      porciones--;
      sLock.unlock();
      obtener_porcion();
      comer();
    }else{
      sLock.unlock();
    }    
  }
}
\end{lstlisting}


\end{columns}

\end{frame}





\end{document}
