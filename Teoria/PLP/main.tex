\documentclass[pdf
]{beamer}

\mode<handout>{}
\usepackage[utf8]{inputenc}
\usepackage{hyperref}
\usepackage{algorithm} 
\usepackage[noend]{algpseudocode}
\usepackage[spanish]{babel}
\usepackage{enumitem} % enum romano
\usepackage{listings}


\title{Prueba de Oposición}
\subtitle{AY2 - Teoría}
\author{\large{Teodoro Freund}}
\date{5 de Noviembre del 2018 \\ \footnotesize{11:00}} %Chequear

\beamertemplatenavigationsymbolsempty

\begin{document}

\floatname{algorithm}{Algoritmo}
\floatname{Procedure}{pro}
\begin{frame}
\titlepage
\end{frame}

\begin{frame}{El Ejercicio 1.19 (retocado)}
Se cuenta con la siguiente representación de conjuntos \texttt{type Conj a = (a->Bool)} caracterizados por su
función de pertenencia. De este modo, si \texttt{c} es un conjunto y \texttt{e} un elemento, la expresión \texttt{c e} devuelve \texttt{True} sii \texttt{e}
pertenece a \texttt{c}.

\pause

\begin{enumerate}[label=(\roman*)]

\item[(III)] Definir un conjunto de funciones que contenga infinitos elementos, pero no todos, y dar su tipo. Además, demuestre (en Haskell) que tiene infinitas funciones pero no todas.

\end{enumerate}

\end{frame}






\begin{frame}{Contexto}
\begin{itemize}
\item Materia: Paradigmas de Lenguajes de Programación
\pause
\item Programación Funcional, \texttt{Haskell}
\pause

\item Primera Parte de la materia:
\begin{itemize}
\item Los alumnos ya retomaron \texttt{Haskell}, reforzando su sintáxis y las herramientas que nos provee.

\item Ya practicaron con esquemas de recursión como \texttt{fold}.

\end{itemize}


\end{itemize}

\end{frame}


\begin{frame}{¿Por qué?}

\begin{itemize}
\item[$\bullet$] Es un ejercicio que ejemplifica muchas cositas de Haskell que no se ven tanto en la materia. Es una buena forma de decir \textit{"Terminamos con Haskell, pero Haskell no terminó"}.

\item[$\bullet$] Da para hablar un montón de temas, entre ellos:
\begin{itemize}
\item[+] la semántica de \texttt{undefined},
\item[+] contravarianza,
\item[+] typeclasses (\texttt{Eq, Foldable}),
\item[+] cosas infinitas y Haskell,
\item[+] etc.
\end{itemize}

\item[$\bullet$] Es un buen ejercicio para dar en clase y analizarlo en profundidad.


\end{itemize}

\end{frame}

\begin{frame}{El Ejercicio 1.19 (retocado), de nuevo}
Se cuenta con la siguiente representación de conjuntos \texttt{type Conj a = (a->Bool)} caracterizados por su
función de pertenencia. De este modo, si \texttt{c} es un conjunto y \texttt{e} un elemento, la expresión \texttt{c e} devuelve \texttt{True} sii \texttt{e}
pertenece a \texttt{c}.


\begin{enumerate}[label=(\roman*)]


\item[(III)] Definir un conjunto de funciones que contenga infinitos elementos, pero no todos, y dar su tipo. Además, demuestre (en Haskell) que tiene infinitas funciones pero no todas.


\end{enumerate}

\end{frame}

%
%\begin{frame}{(I) \texttt{vacio} y \texttt{agregar}}
%\begin{itemize}
%\item \texttt{type Conj a = a -> Bool}
%\item 
%\item \texttt{vacio :: Conj a}
%\pause
%\item \texttt{-- vacio :: a -> Bool}
%\pause
%\item \texttt{vacio =  $\lambda$x -> False}
%\pause
%\item \texttt{vacio x = False}
%\pause
%\item \texttt{vacio \_ = False}
%\pause
%\item \texttt{vacio = const False}
%\pause
%\item 
%\item \texttt{const :: a -> b -> a}
%
%\end{itemize}
%
%\end{frame}
%
%\begin{frame}{(I) \texttt{vacio} y \texttt{agregar}}
%\begin{itemize}
%\item \texttt{type Conj a = a -> Bool}
%\item 
%\item \texttt{agregar :: Eq a => a -> Conj a -> Conj a}
%\pause
%\item \texttt{agregar a c e | e == a = True}
%\item \texttt{\hspace{2.85 cm}| otherwise = c e}
%\pause
%\item \texttt{agregar a c e = e == a || c e}
%
%
%\end{itemize}
%
%\end{frame}
%
%\begin{frame}{(II) \texttt{singleton}}
%
%\begin{itemize}
%
%\item \texttt{type Conj a = a -> Bool}
%\item
%\item \texttt{singleton :: Eq a => a -> Conj a}
%\pause
%\item \texttt{singleton a e = a == e}
%\pause
%\item \texttt{singleton a e = (==) a e}
%\pause
%\item \texttt{singleton = (==)}
%
%\end{itemize}
%
%\end{frame}
%
%\begin{frame}{(III) \texttt{deLista}}
%
%\begin{itemize}
%\item \texttt{type Conj a = a -> Bool}
%\item
%\item \texttt{deLista :: Eq a => [a] -> Conj a}
%\item \texttt{deLista [] = vacio}
%\item \texttt{deLista (x:xs) = agregar x \$ deLista xs}
%\pause
%\item \texttt{deLista xs e = elem e xs}
%\pause
%\item \texttt{deLista = flip elem}
%\end{itemize}
%
%\pause
%\begin{block}{\texttt{Foldable}}
%\texttt{Prelude$>$ :t elem}
%
%\texttt{elem :: (Eq a, Foldable t) => a -> t a -> Bool}
%\end{block}
%
%\end{frame}

\begin{frame}{(III) Conjunto infinito de funciones}

\begin{itemize}
\item \texttt{infinitos :: Conj (Int -> Int)}
\item \texttt{infinitos f = f 0 `mod` 2 == 0}
\item
\item ¿Cuantos elementos tiene? \pause Creemos una lista de infinitos elementos que están.
\pause
% Tiene infinitos elementos, demos un termino en haskell que contenga a todos ello:
%  map const [2,4..], es lazy, no pasa nada
%  el tipo es [Int -> Int]
% No puedo chequear en haskell que todos estos pertencen al cjto
\item ¿Tiene a todos los elementos? ¿Cuantos no tiene? \pause Creemos una lista de inf elementos que no están.
\pause
% Mostremos alguno que no tenga, f = const 1
% De la misma forma, no tiene a infinitos elementos, map const [1,3..]
\item ¿Es decidible? \pause Definamos una función, para la cual no lo pueda decidir.
\pause
% para todo elemento puedo decir si esta o no,
% que pasa con este f x = f $ x+1
%  De repente, tenemos un conjunto no decidible
\item ¿Podemos hacerlo \texttt{Conj (a -> Int)} y \texttt{Conj (Int -> a)}? \pause Cómo genero algún valor de tipo \texttt{a}. \pause Cómo evalúo un valor de tipo \texttt{a}.
\pause
% veamos el tipo de undefined :: a, osea que undefined puede ser lo que queramos
% pero no lo podemos leer, entonces veamos el primer caso, a -> Int,
% a nosotros nos dan una función f de tipo a -> Int, queremos hacerle algo, por lo general, lo unico que le podemos hacer a una función es aplicarla  (por lo general), entonces, le tenemos que dar un valor de tipo a, rapido dame un valor de tipo a, para cualquier a, el unico es undefined, osea, que si f usa su valor (en realidad, si se lo pasa a una funcion que lo evalua) cagamos, porque se me indefine todo, entonces, en algún sentido, f tiene que ser una función que no dependa de su parámetro de entrada (aka constante)
% y que pasa con el segundo, yo tengo una función, que le aplico un Int, y me puede devolver algo de cualquier valor, pensemos, qué me puede devolver f, eso no quiere decir que f elige el valor, sino que yo elijo el valor y f me lo da, para cualquier valor que yo quiera. Entonces, con un argumento similar al anterior, qué me puede devolver f? UNDEFINED, y que puedo hacer yo con undefined? evaluarlo no, entonces deberia retornar un valor que no dependa de mi único parámetro de entrada, que sea constante...
\item ¿Y \texttt{Conj (a -> b)}? \pause Cómo hago todo eso a la vez.
\pause
% Con argumentos similares al anterior, solo le puedo pasar unde
\item ¿Puedo agregarle elementos? 
\end{itemize}

\end{frame}






\end{document}
