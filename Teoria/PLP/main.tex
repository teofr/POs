\documentclass[pdf]{beamer}

\mode<handout>{}
\usepackage[utf8]{inputenc}
\usepackage{hyperref}
\usepackage{algorithm} 
\usepackage[noend]{algpseudocode}
\usepackage[spanish]{babel}
\usepackage{enumitem} % enum romano
\usepackage{listings}
% Code syntax highlighter
\usepackage{minted}


\title{Prueba de Oposición}
\subtitle{AY1 - Simple}
\author{\large{Teodoro Freund}}
\date{10 de Agosto del 2022 \\ \footnotesize{18:00}} %Chequear

\beamertemplatenavigationsymbolsempty

\begin{document}

\floatname{algorithm}{Algoritmo}
\floatname{Procedure}{pro}
\begin{frame}
\titlepage
\end{frame}

\begin{frame}{El Ejercicio 1.20\footnote{En la guía del $1^{er}$ cuatrimestre del 2022} (retocado)}
Se cuenta con la siguiente representación de conjuntos \texttt{type Conj a = (a->Bool)} caracterizados por su
función de pertenencia. De este modo, si \texttt{c} es un conjunto y \texttt{e} un elemento, la expresión \texttt{c e} devuelve \texttt{True} sii \texttt{e}
pertenece a \texttt{c}.

\pause

\begin{enumerate}[label=(\roman*)]

\item[(I)] Definir la constante \texttt{vacio :: Conj a}, y la función \texttt{agregar :: Eq a => a -> Conj a -> Conj a}.

\item[(III)] Definir un conjunto de funciones que contenga infinitos elementos, pero no todos, y dar su tipo. Además, demuestre (con ejemplos en Haskell) que tiene infinitas funciones pero no todas.

\end{enumerate}

\end{frame}


% Se cuenta con la siguiente representación de conjuntos type Conj a = (a->Bool) caracterizados por su
% función de pertenencia. De este modo, si c es un conjunto y e un elemento, la expresión c e devuelve True sii e
% pertenece a c.
% i. Denir la constante vacío :: Conj a, y la función agregar :: Eq a => a -> Conj a -> Conj a.
% ii. Escribir las funciones intersección y unión (ambas de tipo Conj a -> Conj a-> Conj a).
% iii. Denir un conjunto de funciones que contenga innitos elementos, y dar su tipo.
% iv. Denir la función singleton :: Eq a => a -> Conj a, que dado un valor genere un conjunto con ese
% valor como único elemento.
% v. ¾Puede denirse un map para esta estructura? ¾De qué manera, o por qué no?




\begin{frame}{Contexto}
\begin{itemize}
\item Materia: Paradigmas de Lenguajes de Programación
\pause
\item Programación Funcional, \texttt{Haskell}
\pause

\item Primera Parte de la materia:
\begin{itemize}
\item Los alumnos ya retomaron \texttt{Haskell}, reforzando su sintáxis y las herramientas que nos provee.

\item Ya practicaron con esquemas de recursión como \texttt{fold}.

\item Se encuentran atacando problemas más complejos con Haskell, antes de
encarar de lleno los temas más formales de la materia. 

\end{itemize}


\end{itemize}

\end{frame}


\begin{frame}{¿Por qué?}

\begin{itemize}
\item[$\bullet$] Es un ejercicio que ejemplifica muchas cosas de Haskell que no se ven tanto en la materia. Es una buena forma de decir \textit{"Terminamos con Haskell, pero Haskell no terminó"}.

\item[$\bullet$] Si se lo extiende un poco, da para hablar un montón de temas, entre ellos:
\begin{itemize}
\item[+] la semántica de \texttt{undefined},
\item[+] contravarianza,
% \item[+] typeclasses (\texttt{Eq, Foldable}),
\item[+] cosas infinitas y Haskell,
\item[+] etc.
\end{itemize}

\item[$\bullet$] Es un buen ejercicio para dar en clase y analizarlo en profundidad.


\end{itemize}

\end{frame}

\begin{frame}{El Ejercicio 1.20 (retocado), de nuevo}
Se cuenta con la siguiente representación de conjuntos \texttt{type Conj a = (a->Bool)} caracterizados por su
función de pertenencia. De este modo, si \texttt{c} es un conjunto y \texttt{e} un elemento, la expresión \texttt{c e} devuelve \texttt{True} sii \texttt{e}
pertenece a \texttt{c}.


\begin{enumerate}[label=(\roman*)]

\item[(I)] Definir la constante \texttt{vacio :: Conj a}, y la función \texttt{agregar :: Eq a => a -> Conj a -> Conj a}.

\item[(III)] Definir un conjunto de funciones que contenga infinitos elementos, pero no todos, y dar su tipo. Además, demuestre (con ejemplos en Haskell) que tiene infinitas funciones pero no todas.

\end{enumerate}

\end{frame}

%
%\begin{frame}{(I) \texttt{vacio} y \texttt{agregar}}
%\begin{itemize}
%\item \texttt{type Conj a = a -> Bool}
%\item 
%\item \texttt{vacio :: Conj a}
%\pause
%\item \texttt{-- vacio :: a -> Bool}
%\pause
%\item \texttt{vacio =  $\lambda$x -> False}
%\pause
%\item \texttt{vacio x = False}
%\pause
%\item \texttt{vacio \_ = False}
%\pause
%\item \texttt{vacio = const False}
%\pause
%\item 
%\item \texttt{const :: a -> b -> a}
%
%\end{itemize}
%
%\end{frame}
%
%\begin{frame}{(I) \texttt{vacio} y \texttt{agregar}}
%\begin{itemize}
%\item \texttt{type Conj a = a -> Bool}
%\item 
%\item \texttt{agregar :: Eq a => a -> Conj a -> Conj a}
%\pause
%\item \texttt{agregar a c e | e == a = True}
%\item \texttt{\hspace{2.85 cm}| otherwise = c e}
%\pause
%\item \texttt{agregar a c e = e == a || c e}
%
%
%\end{itemize}
%
%\end{frame}
%
%\begin{frame}{(II) \texttt{singleton}}
%
%\begin{itemize}
%
%\item \texttt{type Conj a = a -> Bool}
%\item
%\item \texttt{singleton :: Eq a => a -> Conj a}
%\pause
%\item \texttt{singleton a e = a == e}
%\pause
%\item \texttt{singleton a e = (==) a e}
%\pause
%\item \texttt{singleton = (==)}
%
%\end{itemize}
%
%\end{frame}
%
%\begin{frame}{(III) \texttt{deLista}}
%
%\begin{itemize}
%\item \texttt{type Conj a = a -> Bool}
%\item
%\item \texttt{deLista :: Eq a => [a] -> Conj a}
%\item \texttt{deLista [] = vacio}
%\item \texttt{deLista (x:xs) = agregar x \$ deLista xs}
%\pause
%\item \texttt{deLista xs e = elem e xs}
%\pause
%\item \texttt{deLista = flip elem}
%\end{itemize}
%
%\pause
%\begin{block}{\texttt{Foldable}}
%\texttt{Prelude$>$ :t elem}
%
%\texttt{elem :: (Eq a, Foldable t) => a -> t a -> Bool}
%\end{block}
%
%\end{frame}


\begin{frame}[fragile]{\texttt{vacio :: Conj a}}

\begin{overprint}
    
\onslide<1>
\begin{block}{Pizarrón/Editor}
\begin{minted}{haskell}
vacio :: Conj a

\end{minted}
\end{block}

\onslide<2>
\begin{block}{Pizarrón/Editor}
\begin{minted}{haskell}
vacio :: Conj a
vacio = \e ->
\end{minted}
\end{block}

\onslide<3>
\begin{block}{Pizarrón/Editor}
\begin{minted}{haskell}
vacio :: Conj a
vacio = \e -> False
\end{minted}
\end{block}

\onslide<4>
\begin{block}{Pizarrón/Editor}
\begin{minted}{haskell}
vacio :: Conj a
vacio _ = False
\end{minted}
\end{block}

\end{overprint}

\begin{itemize}
    \item<1-> Sabemos que el tipo del conjunto tiene que ser de \texttt{a}, donde \texttt{a} puede ser cualquier tipo
    \item<2-> Entonces, empecemos a programarlo, y seguro tiene que tomar un elemento y hacer algo con él
    \item<3-> Si bien es fácil ver que debería retornar siempre \texttt{False}, vale la pena analizar que otra cosa podríamos
    hacer con un elemento sobre el cual no sabemos nada sobre su tipo, y llegar a la conclusión de que en realidad no podemos hacer casi nada
    \item<4-> Por último, para ser idiomatico con Haskell, podemos extrar el lambda y cambiarlo por un \texttt{\_} ya que no se usa
\end{itemize}

\end{frame}


\begin{frame}[fragile]{\texttt{agregar :: Eq a => a -> Conj a -> Conj a}}

\begin{overprint}
    
\onslide<1>
\begin{block}{Pizarrón/Editor}
\begin{minted}{haskell}
agregar :: Eq a => a -> Conj a -> Conj a

\end{minted}
\end{block}

\onslide<2>
\begin{block}{Pizarrón/Editor}
\begin{minted}{haskell}
agregar :: Eq a => a -> Conj a -> Conj a
agregar a c = \e -> 
\end{minted}
\end{block}

\onslide<3>
\begin{block}{Pizarrón/Editor}
\begin{minted}{haskell}
agregar :: Eq a => a -> Conj a -> Conj a
agregar a c = \e -> (e == a) || c e
\end{minted}
\end{block}

\onslide<4->
\begin{block}{Pizarrón/Editor}
\begin{minted}{haskell}
agregar :: Eq a => a -> Conj a -> Conj a
agregar a c e = (e == a) || c e
\end{minted}
\end{block}

\end{overprint}

% \begin{itemize}
%     \item<1-> Sabemos que el tipo del conjunto tiene que ser de \texttt{a}, donde \texttt{a} puede ser cualquier tipo
%     \item<2-> Entonces, empecemos a programarlo, y seguro tiene que tomar un elemento y hacer algo con él
%     \item<3-> Si bien es fácil ver que debería retornar siempre \texttt{False}, vale la pena analizar que otra cosa podríamos
%     hacer con un elemento sobre el cual no sabemos nada sobre su tipo, y llegar a la conclusión de que en realidad no podemos hacer casi nada
%     \item<4-> Por último, para ser idiomatico con Haskell, podemos extrar el lambda y cambiarlo por un \texttt{\_} ya que no se usa
% \end{itemize}

\begin{block}<5->{Pequeña conclusión}
Toda esta primer parte nos permite, no solo entender como utilizar las funciones como tipos de datos (objetivo de la práctica),
pero además programar ejercicios en Haskell un poco más complejos, y empezar a mostrar detalles no centrales de la materia pero importantes, como buenas prácticas, formas idiomáticas, etc.
\end{block}

\end{frame}




\begin{frame}[fragile]{Conjunto infinito de funciones}

\begin{overprint}
    
\onslide<1>
\begin{block}{Pizarrón/Editor}
\begin{minted}{haskell}
infinitos :: Conj (a -> b)
\end{minted}
\end{block}

\onslide<2-3>
\begin{block}{Pizarrón/Editor}
\begin{minted}{haskell}
infinitos :: Conj (Bool -> Bool)
\end{minted}
\end{block}

\onslide<4>
\begin{block}{Pizarrón/Editor}
\begin{minted}{haskell}
infinitos :: Conj (Int -> Int)
\end{minted}
\end{block}

\onslide<5>
\begin{block}{Pizarrón/Editor}
\begin{minted}{haskell}
infinitos :: Conj (Int -> Int)
infinitos f = 
\end{minted}
\end{block}

\onslide<6>
\begin{block}{Pizarrón/Editor}
\begin{minted}{haskell}
infinitos :: Conj (Int -> Int)
infinitos f = True
\end{minted}
\end{block}

\onslide<7>
\begin{block}{Pizarrón/Editor}
\begin{minted}{haskell}
infinitos :: Conj (Int -> Int)
infinitos f = f
\end{minted}
\end{block}

\onslide<8,9>
\begin{block}{Pizarrón/Editor}
\begin{minted}{haskell}
infinitos :: Conj (Int -> Int)
infinitos f = f 0
\end{minted}
\end{block}

\onslide<10>
\begin{block}{Pizarrón/Editor}
\begin{minted}{haskell}
infinitos :: Conj (Int -> Int)
infinitos f = f 0 `mod` 2 == 0
\end{minted}
\end{block}

\end{overprint}

\begin{itemize}
    \item<1-> Sabemos que el tipo del conjunto tiene que ser un conjunto de funciones
    \item<2-> Podemos probar hacerlo de booleanos en booleanos, pero esto puede traer un problema
    \item<3-> Entonces, de minima tiene que ser un tipo infinito, probemos con alguno conocido, por ejemplo Int
    \item<5-> Empecemos a programarla\onslide<6->{, probemos uno que devuelva constantemente True}
    \item<7-> Por lo general, teniendo una función, la única opción es aplicarla\onslide<8->{, en este caso a un Int, como 0}
    \item<9-> Ahora, tenemos el resultado que es un Int, busquemos alguna propiedad para la cual haya infinitos Trues e infinitos Falses
    \onslide<10->{, por ejemplo, podemos discernir según la paridad}
\end{itemize}


    
\end{frame}


\begin{frame}[fragile]{Ejemplo de infinitos elementos}


\begin{overprint}
    
\onslide<1>
\begin{block}{Pizarrón/Editor}
\begin{minted}{haskell}
ejemplo :: [Int -> Int]

\end{minted}
\end{block}

\onslide<2>
\begin{block}{Pizarrón/Editor}
\begin{minted}{haskell}
ejemplo :: [Int -> Int]
ejemplo = [ \x -> if (x == 0) then 2 else 1 ]
\end{minted}
\end{block}

\onslide<3>
\begin{block}{Pizarrón/Editor}
\begin{minted}{haskell}
ejemplo :: [Int -> Int]
ejemplo = [ \x -> 2, \x -> 4, \x -> 6 ]
\end{minted}
\end{block}

\onslide<4>
\begin{block}{Pizarrón/Editor}
\begin{minted}{haskell}
ejemplo :: [Int -> Int]
ejemplo = [ const 2, const 4, const 6 ]
\end{minted}
\end{block}

\onslide<5>
\begin{block}{Pizarrón/Editor}
\begin{minted}{haskell}
ejemplo :: [Int -> Int]
ejemplo = map const [2,4..]
\end{minted}
\end{block}

\end{overprint}

\begin{itemize}
    \item<1-> Sabemos que el tipo del conjunto tiene que ser un conjunto de funciones
    \item<2-> Y tenemos que crear una lista de funciones que al ser aplicadas a 0, retornen un número par
    \onslide<3->{, para facilitar el ejercicio, podemos hacer que devuelvan siempre el mismo valor.}
    \onslide<4->{Y un poco más idiomaticamente...}
    \item<5-> Y para hacerla infinita podemos usar alguna de las muchas facilidades que Haskell
    nos provee para manejar listas infinitas
\end{itemize}


\end{frame}

\begin{frame}[fragile]{Ejemplo de infinitos no-elementos}



\begin{block}{Pizarrón/Editor}
\begin{minted}{haskell}
contraejemplo :: [Int -> Int]
contraejemplo = map const [1,3..]
\end{minted}
\end{block}


\end{frame}

\begin{frame}[fragile]{Decibilidad de la pertenencia al conjunto}

\begin{overprint}
    
\onslide<1>
\begin{block}{Pizarrón/Editor}
\begin{minted}{haskell}
indecidible :: Int -> Int
indecidible x = 
\end{minted}
\end{block}

\onslide<2>
\begin{block}{Pizarrón/Editor}
\begin{minted}{haskell}
indecidible :: Int -> Int
indecidible x = indecidible $ x + 1
\end{minted}
\end{block}

\onslide<3>
\begin{block}{Pizarrón/Editor}
\begin{minted}{haskell}
indecidible :: Int -> Int
indecidible x = indecidible $ x + 1

perteneceindecidible = infinitos indecidible
\end{minted}
\end{block}

\end{overprint}

\begin{itemize}
    \item<1-> Busquemos alguna funcion que sea indecidible verificar su pertenencia
    \item<2-> Una opción es alguna función que nunca termine, de tal forma de nunca poder saber el resultado de aplicarla a 0
    \item<3-> Apenas infinitos intente aplicarla a 0 para verificar su pertenencia se va a colgar
\end{itemize}


\end{frame}

\begin{frame}{Fin}
\large
¿Preguntas?
\end{frame}


% \begin{frame}{(III) Conjunto infinito de funciones}

% \begin{itemize}
% \item \texttt{infinitos :: Conj (Int -> Int)}
% \item \texttt{infinitos f = f 0 `mod` 2 == 0}
% \item
% \item ¿Cuantos elementos tiene? \pause Creemos una lista de infinitos elementos que están.
% \pause
% % Tiene infinitos elementos, demos un termino en haskell que contenga a todos ello:
% %  map const [2,4..], es lazy, no pasa nada
% %  el tipo es [Int -> Int]
% % No puedo chequear en haskell que todos estos pertencen al cjto
% \item ¿Tiene a todos los elementos? ¿Cuantos no tiene? \pause Creemos una lista de inf elementos que no están.
% \pause
% % Mostremos alguno que no tenga, f = const 1
% % De la misma forma, no tiene a infinitos elementos, map const [1,3..]
% \item ¿Es decidible? \pause Definamos una función, para la cual no lo pueda decidir.
% \pause
% % para todo elemento puedo decir si esta o no,
% % que pasa con este f x = f $ x+1
% %  De repente, tenemos un conjunto no decidible
% \item ¿Podemos hacerlo \texttt{Conj (a -> Int)} y \texttt{Conj (Int -> a)}? \pause Cómo genero algún valor de tipo \texttt{a}. \pause Cómo evalúo un valor de tipo \texttt{a}.
% \pause
% % veamos el tipo de undefined :: a, osea que undefined puede ser lo que queramos
% % pero no lo podemos leer, entonces veamos el primer caso, a -> Int,
% % a nosotros nos dan una función f de tipo a -> Int, queremos hacerle algo, por lo general, lo unico que le podemos hacer a una función es aplicarla  (por lo general), entonces, le tenemos que dar un valor de tipo a, rapido dame un valor de tipo a, para cualquier a, el unico es undefined, osea, que si f usa su valor (en realidad, si se lo pasa a una funcion que lo evalua) cagamos, porque se me indefine todo, entonces, en algún sentido, f tiene que ser una función que no dependa de su parámetro de entrada (aka constante)
% % y que pasa con el segundo, yo tengo una función, que le aplico un Int, y me puede devolver algo de cualquier valor, pensemos, qué me puede devolver f, eso no quiere decir que f elige el valor, sino que yo elijo el valor y f me lo da, para cualquier valor que yo quiera. Entonces, con un argumento similar al anterior, qué me puede devolver f? UNDEFINED, y que puedo hacer yo con undefined? evaluarlo no, entonces deberia retornar un valor que no dependa de mi único parámetro de entrada, que sea constante...
% \item ¿Y \texttt{Conj (a -> b)}? \pause Cómo hago todo eso a la vez.
% \pause
% % Con argumentos similares al anterior, solo le puedo pasar unde
% \item ¿Puedo agregarle elementos? 
% \end{itemize}

% \end{frame}






\end{document}
