\documentclass[presentation
]{beamer}

\mode<handout>{}
\usepackage[utf8]{inputenc}
\usepackage{hyperref}
\usepackage{algorithm} 
\usepackage[noend]{algpseudocode}
\usepackage[spanish]{babel}


\title{Prueba de Oposición}
\subtitle{AY2 - Teoría}
\author{\large{Teodoro Freund}}
\date{19 de Septiembre del 2017 \\ \footnotesize{14:20}} %Chequear

\beamertemplatenavigationsymbolsempty

\begin{document}

\floatname{algorithm}{Algoritmo}
\floatname{Procedure}{pro}
\begin{frame}
\titlepage
\end{frame}



\begin{frame}{El Ejercicio 3}

Dado $L = \{~0^i1^j ~| ~i>j \lor i~ par\}$

\begin{itemize}
\item[a] Demostrar que L cumple: 
$$ \forall \alpha (\alpha \in L \land |\alpha| \geq 2 \Rightarrow \exists x, y, z $$ 
$$  (\alpha = xyz \land |xy| \leq 2 \land |y| \geq 1 \land \forall k (xy^kz \in L))) $$

\item[b] Demostrar que L no es regular
\end{itemize}

 

\end{frame}


\begin{frame}{Contexto}
\begin{itemize}
\item Materia: Teoría de Lenguajes
\pause
\item Lenguajes regulares, Lema de Pumping
\pause

\item Primera Parte de la materia:
\begin{itemize}
\item Los alumnos vieron los conceptos de gramáticas regulares, expresiones regulares y automátas. Y entienden su equivalencia.
\item Se dio el concepto de lenguaje regular. Y se vieron propiedades sobre estos.
\item Se vio el Lema de Pumping como herramienta para probar no regularidad.
\end{itemize}


\end{itemize}

\end{frame}

\begin{frame}{¿Por qué?}

\begin{itemize}
\item Es un problema que integra distintos conocimientos sobre lenguajes regulares y propiedades sobre ellos.

\item Podría formar parte de una guía de ejercicios e incluso ser resuelto en clase.

\item Deja asentado que si $L$ es un lenguaje, no vale en general:

\begin{center}

$L$ cumple el Lema de Pumping $\Rightarrow$ $L$ es regular
\end{center}


\end{itemize}

\end{frame}

\begin{frame}{El Ejercicio 3, de nuevo}
Dado $L = \{~0^i1^j ~| ~i>j \lor i~ par\}$

\begin{itemize}
\item[a] Demostrar que L cumple: 
$$ \forall \alpha (\alpha \in L \land |\alpha| \geq 2 \Rightarrow \exists x, y, z $$ 
$$  (\alpha = xyz \land |xy| \leq 2 \land |y| \geq 1 \land \forall k (xy^kz \in L))) $$

\item[b] Demostrar que L no es regular
\end{itemize}

\end{frame}

\begin{frame}{a. Demostración por casos}
$L = \{~0^i1^j ~| ~i>j \lor i~ par\}$, queremos ver que L cumple:

$$ \forall \alpha (\alpha \in L \land |\alpha| \geq 2 \Rightarrow \exists x, y, z $$ 
$$  (\alpha = xyz \land |xy| \leq 2 \land |y| \geq 1 \land \forall k (xy^kz \in L))) $$


Primero notemos que $\alpha = 0^i1^j$ y que $i+j \geq 2$.

\pause

\begin{itemize}
\item $i$ par:
\pause
\begin{itemize}
\item $i = 0$: \emph{Pizarrón}
\pause
\item $i \neq 0 \Rightarrow i\geq 2$: \emph{Pizarrón}
\end{itemize}
\pause
\item $i$ impar $\Rightarrow i >j$: \emph{Pizarrón}
\end{itemize}
\end{frame}


\begin{frame}{b. Por el absurdo}

Primero que nada recordemos los siguientes resultados:

\begin{itemize}
\item Si r es una expresión regular, el lenguaje que genera es regular.
\item Sean $L$ y $K$ dos lenguajes regulares:
\begin{itemize}
\item $\overline{L}$ es regular. Esto se piensa con $\Sigma^*$ cómo nuestro universo y $\overline{L} = \Sigma^* - L$.
\item $L \cap K$ es un lenguaje regular
\item $\exists n $ tal que se cumple el Lema de Pumping.
\end{itemize}
\end{itemize}

\pause

Galerazo:
\begin{itemize}
\item Sea $R$ el lenguaje regular generado por la siguiente expresión regular: $(00)^*1^*$
\pause 
\item Supongamos que $L$ es regular
\pause
\item y llamemos $K$ al lenguaje regular dado por $K = L\cap \overline{R} = L - R$
\end{itemize}



\end{frame}




\end{document}
